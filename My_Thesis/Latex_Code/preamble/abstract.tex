
\thispagestyle{empty}

\begin{center}
{\Large \textbf{Abstract}}\\%[1.5cm]
\end{center}

\setlength{\parindent}{0em}
We head out to develop a FPGA system around AJIT cores and integrating HPC peripherals designed though inhouse and industry standard HLS
tools into this system to increase performance of computationally expensive operations such as convolution of images. The goal of this
project is to build a FPGA system capable of booting a custom built OS on the AJIT core. In the first part of the document we focus on the
development cycle of the FPGA system where we first describe the processor itself, detailed need for such a system, our setup choices, a
telescopic system view, individual block designs and respective design choices made, and the test procedures for the system.

In second part of the document we start the discussion on development of a Convolution core using the inhouse HLS tools. Here we discuss the
the need of such peripherals which we later on justify through performance metrics also. Later on we discuss how we began designing the
peripheral, the design of the internal architecture and subsystems, the choice of design tools and also our test setup. Finally we compare
the hardware resource usages and timing results from different variants of the generated convolution engine.

\afterpage{\blankpage}
\clearpage
