\usepackage[includehead, left=1in, right=1in, top=0.8in, bottom=0.8in]{geometry}
%\usepackage[framemethod=TikZ]{mdframed}
\usepackage{graphicx}
\usepackage{xcolor,colortbl}
\usepackage{listings} % Allows code-listings
%\usepackage{wrapfig}
\usepackage{courier} % proper monospace font for code
%\usepackage{lscape}
\usepackage{rotating}
\usepackage{epstopdf}
\usepackage{hyperref} %linking contents  table
\usepackage{xstring} %string-manipulation
\usepackage{enumitem} %used for enumerate-manipulation
\usepackage{float} %used to properly place float-objects (figures)
\usepackage[T1]{fontenc}
\usepackage{titlesec, blindtext, color}
%\usepackage[norsk]{babel}
\usepackage[utf8]{inputenc}
\usepackage{tabularx,ragged2e,booktabs,caption}
\usepackage{subcaption}
\usepackage{ulem} %used for strikeout
\usepackage{verbatim} %used for block-commenting
\usepackage[toc,page]{appendix}
\usepackage{url}
\usepackage{amsmath}
\usepackage{algorithm}
\usepackage[noend]{algpseudocode}
\usepackage{blindtext}
\usepackage{afterpage}
\usepackage{setspace}
\usepackage{csquotes}
\usepackage[nottoc]{tocbibind}
\usepackage{titlesec}
\usepackage{siunitx}
\usepackage{tikz} % To generate the plot from csv
\usepackage{pgfplots}
\usepackage{csvsimple}

\pgfplotsset{compat=newest} % Allows to place the legend below plot
\usepgfplotslibrary{units} % Allows to enter the units nicely

\sisetup{
  round-mode          = places,
  round-precision     = 2,
}

\setcounter{secnumdepth}{4}

%\titleformat{\paragraph}
%{\normalfont\normalsize\bfseries}{\theparagraph}{1em}{}
%\titlespacing*{\paragraph}
%{0pt}{3.25ex plus 1ex minus .2ex}{1.5ex plus .2ex}

%\setlength{\headsep}{0in}

\newcommand\blankpage{%
    \null
    \thispagestyle{empty}%
    \addtocounter{page}{-1}%
    \newpage}

\newcommand{\customblankpage}{\afterpage{\blankpage} \clearpage}

%%%%%%%%%%%%%%%%%%%%%%%%%%%%%%%%%%%%%%%%%%%%%%%%%%%%%%%%%%%%%% 
%% Start of fancyhdr settings. 
%%%%%%%%%%%%%%%%%%%%%%%%%%%%%%%%%%%%%%%%%%%%%%%%%%%%%%%%%%%%%% 

\usepackage{fancyhdr}
\pagestyle{fancy}
\fancyhead{}
%\renewcommand{\chaptermark}[1]{ \markboth{#1}{} }
%\renewcommand{\sectionmark}[1]{ \markright{\thesection \hspace{1em} #1}{}}

\fancyhead{} % clear old format
\lhead{\textit{\leftmark}}
\rhead{\thesection}
\cfoot{\thepage}
\renewcommand{\headrulewidth}{0.4pt}
\renewcommand{\footrulewidth}{0.4pt}

%%%%%%%%%%%%%%%%%%%%%%%%%%%%%%%%%%%%%%%%%%%%%%%%%%%%%%%%%%%%%% 
%% Below fancyhdr settings has been added from a SO answer.
%%%%%%%%%%%%%%%%%%%%%%%%%%%%%%%%%%%%%%%%%%%%%%%%%%%%%%%%%%%%%% 

%\fancyhead{}
%\fancyhead[R]{\footnotesize
%  \begin{tabular}[b]{@{}r@{}}
%    \nouppercase{\leftmark}\\[3pt]
%    \nouppercase{\rightmark}
%  \end{tabular}%
%}
%\setlength{\headheight}{22pt}

%%%%%%%%%%%%%%%%%%%%%%%%%%%%%%%%%%%%%%%%%%%%%%%%%%%%%%%%%%%%%% 
%% End of fancyhdr settings. 
%%%%%%%%%%%%%%%%%%%%%%%%%%%%%%%%%%%%%%%%%%%%%%%%%%%%%%%%%%%%%% 

\definecolor{gray75}{gray}{0.75}
\definecolor{gray1}{gray}{0.97}
\definecolor{gray2}{gray}{0.90}
\definecolor{gray3}{gray}{0.80}
\definecolor{gray4}{gray}{0.63}

\lstset{
    basicstyle=\ttfamily, % Standardschrift
    numbers=left,               % Ort der Zeilennummern
    numberstyle=\tiny,          % Stil der Zeilennummern
    aboveskip= 0pt,
    emphstyle=\textbf,
    escapechar=ä,
%    numbersep=5pt,              % Abstand der Nummern zum Text
%    tabsize=2,                  % Groesse von Tabs
%    extendedchars=true,         %
%    breaklines=true,            % Zeilen werden Umgebrochen
%    keywordstyle=\color{red},
    frame=single,         
    otherkeywords={0a:00\.0, \$, 128M, 32-bit, Xilinx, 7038,134217728 }
  %  keywordstyle=[1]\textbf,    % Stil der Keywords
  %  keywordstyle=[2]\textbf,    %
  %  keywordstyle=[3]\textbf,    %
  %  keywordstyle=[4]\textbf,   \sqrt{\sqrt{}} %
  %  stringstyle=\color{white}\ttfamily, % Farbe der String
%    showspaces=false,           % Leerzeichen anzeigen ?
%    showtabs=false,             % Tabs anzeigen ?
%    xleftmargin=17pt,
%    framexleftmargin=17pt,
%    framexrightmargin=5pt,
%    framexbottommargin=4pt,
%    %backgroundcolor=\color{lightgray},
%    showstringspaces=false      % Leerzeichen in Strings anzeigen ?        
}

%\DeclareCaptionFont{blue}{\color{blue}} 

%\captionsetup[lstlisting]{singlelinecheck=false, labelfont={blue}, textfont={blue}}
%\usepackage{caption}
%\DeclareCaptionFont{white}{\color{white}}
%\DeclareCaptionFormat{listing}{\colorbox[cmyk]{0.43, 0.35, 0.35,0.01}{\parbox{\textwidth}{\hspace{15pt}#1#2#3}}}
%\captionsetup[lstlisting]{format=listing,labelfont=white,textfont=white, singlelinecheck=false, margin=0pt, font={bf,footnotesize}}

\newcommand{\hsp}{\hspace{20pt}}
\newcommand{\HRule}{\rule{\linewidth}{0.5mm}}

\newcommand{\img}[2]{ %images with borders
	\begin{figure}[H]
	\centering
	\fcolorbox{black}{black}{\includegraphics[width=14cm]{img/#1}}
	\caption[#2]{#2}
	\end{figure}
}

\newcommand{\imgC}[3]{ %images with borders and custom parameters
	\begin{figure}[H]
	\centering
	\fcolorbox{black}{black}{\includegraphics[#2]{img/#1}}
	\caption[#3]{#3}
	\end{figure}
}

\newcommand{\plotfromcsv}[2]{
\begin{figure}[H]
  \begin{center}
    \begin{tikzpicture}
      \begin{axis}[
          width=0.8\textwidth, % Scale the plot to \linewidth
          grid=major, % Display a grid
          grid style={dashed,gray!30}, % Set the style
          xlabel=Number of Cores, % Set the labels
          ylabel=Flip Flops,
          %x unit=\si{\units}, % Set the respective units
          %y unit=\si{\ampere},
          legend style={at={(0.7,0.3)},anchor=north}, % Put the legend below the plot
          %x tick label style={rotate=90,anchor=east} % Display labels sideways
        ]
        \addplot 
        % add a plot from table; you select the columns by using the actual name in
        % the .csv file (on top)
        table[x=cores,y=ffs,col sep=comma] {#1}; 
        \legend{#2}
      \end{axis}
    \end{tikzpicture}
    \caption{Number of Cores Vs Hardware resources}
  \end{center}
\end{figure}
}

\newcommand{\liste}[1]{\begin{enumerate}[label=\textbf{#1:\arabic*}]}
\newcommand{\inn}{\begin{enumerate}[label*=\textbf{.\arabic*}]}
\newcommand{\ut}{\end{enumerate}}

\newcommand{\refer}[5]{\bibitem {#1} #2. "#3" \textit{#4} (#5).}

%\newcommand{\labs}{\label{sec:utdypning}}
%\newcommand{\labl}[1]{\label{itm:#1}}
%\newcommand{\refs}{(se side~\pageref{sec:utdypning})}
%\newcommand{\refl}[1]{\ref{itm:#1} - }


%\parindent=5pt
%\baselineskip=0pt
%\parskip=0pt

\hypersetup{ %Brukes for � fjerne farger fra linker i content table
    colorlinks,
    citecolor=black,
    filecolor=black,
    linkcolor=black,
    urlcolor=black
}

\titleformat{\chapter}[hang]{\Huge\bfseries}{\thechapter\hsp\textcolor{gray75}{|}\hsp}{0pt}{\Huge\bfseries}

