\chapter{Performance of the System}

%\section{Timing and utilization summary}

\section*{Timing Summary}

In the following table we report the timing summary of the synthesized and implemented design of the FPGA System on VC709 operating at
100MHz.

\begin{table}[H]
\centering
\begin{tabular}{c | c | c | c}
\hline
Parameter & Description & Tolerance & Value Obtained \\
\hline
WNS & Worst Negative Slack & 1-2 percent of 10ns & -0.1ns\\
CP & Critical Path & N/A & Exists within Xilinx Smartconnect  
\end{tabular}
\caption{Timing Summary}
\end{table}

\section*{Utilization Summary}
In the following table we report the utilization summary of the synthesized and implemented design of different variants of the FPGA System
on VC709.

\begin{table}[H]
\centering
\begin{tabular}{c | c | c}
\hline
Design & Description & Value Obtained \\
\hline
\end{tabular}
\caption{Utilization Summary}
\end{table}

\section*{Host to FPGA system}

\paragraph{Writing speed of PCIe\\}
\setlength{\parindent}{0em}
Writing speed of PCIe with 6 blocks of 128MB each after preliminary tests was found to be around 400 MB/s which is significantly higher than
the required speed in the current situation and also when compared to the past FPGA system. This speed is really helpful when we need to
write huge amount of data to memory for example to write the image of the custom compiled operating system for AJIT.

\paragraph{Reading speed of PCIe\\}

Reading speed of PCIe with 6 blocks of 128MB each after preliminary tests was found to be around 10 MB/s which is significantly higher than
the required speed in the current situation and also when compared to the past FPGA system. 
