\chapter{Need for a FPGA system}

The FPGA system developed around AJIT Core consists of the necessary peripherals which enable the user to boot a decently packed embedded OS
on it. A flexible FPGA system is preferred where the integration of new peripherals is as smooth as possible by keeping the disturbance
caused to the current configuration as minimal as possible. Here in this system adding a new processor core is as smooth as adding a new
peripheral which is a huge advantage over traditional systems based around a single core, this makes the transition to a multiprocessor
model really easy and convenient.

The above system acts as a test-bed for various purposes such as:-

\begin{itemize}

\item Easy custom OS booting cycles : The system can be easily loaded with custom operating systems compiled for AJIT from scratch or through
embedded OS build systems such as Buildroot or Yocto.

%\item Loading packed embedded operating systems : Larger ob-board DRAM allows us to load heavier OS packed with useful drivers.

\item Debugging of Bootloader related bugs : The FPGA system provides the boot log over serial(UART) to the host machine which is really
helpful in debugging bugs related to initialization of hardware, loading of modules and is also helpful in providing information necessary
in order to optimize to boot time for the system to the maximum extent.

\item Debugging of Driver related bugs : Loading and unloading of kernel modules, memory inspection from host machine directly without going
through the on-board processor channel.

\item Peripheral testing and performance measurement : Any AXI peripheral can be tested directly from the host machine without interfering
with the normal working of the system with proper care for isolation obviously.

\item Faster bootstrapping for custom projects : The FPGA system is highly configurable and is designed to offer basic building blocks and
their control interfaces and respective drivers for their easy integration and development of minimum working example systems. 

\end{itemize}
