\chapter{Hardware Usage comparison}

\section{Background}

Here we compare the hardware resources required by the HDL design generated by Vivado HLS and Aa HLS implementation of the same core.
Also within the Aa HLS implementation we compare the hardware resources required by the Pipelined version of the core with the Non-pipelined
version of the core by varying the number of coprocessors utilized in development of the internal architecture of the core. The default
pipeline depth is currently kept at 7 to produce the following values and plots.

%\section{Vivado HLS Implementation}
%\blindtext

\section{AHIR HLS Implementation}

Here we produce the results from the synthesis and implementation of different variants of our convolution engine where we consider the
number of Flip Flops in the implemented design as our metric to measure the hardware usage of each variant. We produce pipelined and
non-pipelined variants for a given number of coprocessors through two methods namely C HLS and Aa HLS(described in previous chapter).
Through this process we aim to show the advantage of generating a hardware system through Aa HLS instead of C HLS.The board chosen for this
process was Xilinx's VC709.

%\pagebreak

\subsection{C to VHDL}

\subsubsection*{Non-pipelined Version}

\begin{table}[H]
\centering
\begin{tabular}{c|c}%
    \hline
    \bfseries Number of Cores & \bfseries FFs\\\hline % specify table head
    \csvreader[head to column names]{csvs/c_np.csv}{}% use head of csv as column names
    {\\\cores & \ffs} % specify your coloumns here
\end{tabular}
\caption{Hardware Resources}
\end{table}

\plotfromcsv{csvs/c_np.csv}{Non-pipelined Coprocessor}

\subsubsection*{Pipelined Version}

\begin{table}[H]
\centering
\begin{tabular}{c|c}%
    \hline
    \bfseries Number of Cores & \bfseries FFs\\\hline % specify table head
    \csvreader[head to column names]{csvs/c_p.csv}{}% use head of csv as column names
    {\\\cores & \ffs} % specify your coloumns here
\end{tabular}
\caption{Hardware Resources}
\end{table}

\plotfromcsv{csvs/c_p.csv}{Pipelined Coprocessor}

\subsection{Aa to VHDL}

\subsubsection*{Non-pipelined Version}

\begin{table}[H]
\centering
\begin{tabular}{c|c}%
    \hline
    \bfseries Number of Cores & \bfseries FFs\\\hline % specify table head
    \csvreader[head to column names]{csvs/aa_np.csv}{}% use head of csv as column names
    {\\\cores & \ffs} % specify your coloumns here
\end{tabular}
\caption{Hardware Resources}
\end{table}

\plotfromcsv{csvs/aa_np.csv}{Non-pipelined Coprocessor}

\subsubsection*{Pipelined Version}

\begin{table}[H]
\centering
\begin{tabular}{c|c}%
    \hline
    \bfseries Number of Cores & \bfseries FFs\\\hline % specify table head
    \csvreader[head to column names]{csvs/aa_p.csv}{}% use head of csv as column names
    {\\\cores & \ffs} % specify your coloumns here
\end{tabular}
\caption{Hardware Resources}
\end{table}

\plotfromcsv{csvs/aa_p.csv}{Pipelined Coprocessor}
