\chapter{Results}

\section{Timing Results}

Timing results from different variants of the engine design for convoluting an image of size 1024$\times$1024 with a kernel of size
3$\times$3. 

\subsection{C to VHDL}

\subsubsection*{Non-pipelined Version}

\begin{table}[H]
\centering
\begin{tabular}{c|c}%
    \hline
    \bfseries Number of Cores & \bfseries Time Taken(in ns)\\\hline % specify table head
    \csvreader[head to column names]{csvs/c_np_timing.csv}{}% use head of csv as column names
    {\\\cores & \timing} % specify your coloumns here
\end{tabular}
\caption{Timing analysis}
\end{table}

\plottimingfromcsv{csvs/c_np_timing.csv}{Non-pipelined Coprocessor}

\subsubsection*{Pipelined Version}

\begin{table}[H]
\centering
\begin{tabular}{c | c}
\hline
Number of Cores & FFs \\
\hline
1 & 59.640203597 \\
2 & 59.607970220 \\
4 & 59.611384216 \\
8 & 59.690874289 \\
16 & 59.722889924\\
32 & 59.722889924
\end{tabular}
\caption{Timing Results}
\end{table}

\subsection{Aa to VHDL}

\subsubsection*{Non-pipelined Version}

\begin{table}[H]
\centering
\begin{tabular}{c | c}
\hline
Number of Cores & FFs \\
\hline
1 & 59.640203597 \\
2 & 59.607970220 \\
4 & 59.611384216 \\
8 & 59.690874289 \\
16 & 59.722889924\\
32 & 59.722889924
\end{tabular}
\caption{Timing Results}
\end{table}

\subsubsection*{Pipelined Version}

\begin{table}[H]
\centering
\begin{tabular}{c | c}
\hline
Number of Cores & FFs \\
\hline
1 & 59.640203597 \\
2 & 59.607970220 \\
4 & 59.611384216 \\
8 & 59.690874289 \\
16 & 59.722889924\\
32 & 59.722889924
\end{tabular}
\caption{Timing Results}
\end{table}

